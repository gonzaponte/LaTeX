
%Tipo de documento
\documentclass[10pt]{beamer}

%Paquetes
\usepackage{beamerthemesplit}
\usepackage[spanish]{babel}
\usepackage[utf8]{inputenc}
\usepackage{bm} %letras griegas en negrita (uso: \bm{\alpha})
\usepackage{graphicx} %paquete de imágenes
\usepackage{ae,aecompl}
\usepackage{amsmath,amsfonts,latexsym,cancel,color,textcomp,anysize,amsthm,multicol}

%Tema
\usetheme{Marburg}

%Espaciados
%\marginsize{2cm}{1.5cm}{2cm}{1.5cm} %MÁRGENES: izq,der,sup,inf
%\parindent=8mm %sangría 
%\parskip=4mm %espacio entre párrafos

%Cabeceiras
%\lhead{Fermións de Majorana} %CABECERA EN EL CENTRO RESTO DE PÁGINAS
%\rhead{\rightmark}
%\renewcommand{\headrulewidth}{0.4pt}

%Comandos novos
\newcommand{\beq}{\begin{equation}}
\newcommand{\eeq}{\end{equation}}
\newcommand{\bea}{\begin{eqnarray}}
\newcommand{\eea}{\end{eqnarray}}
\newcommand{\bi}{\begin{itemize}}
\newcommand{\ei}{\end{itemize}}
\newcommand{\bc}{\begin{cases}}
\newcommand{\ec}{\end{cases}}
\newcommand{\bbx}{\begin{boxed}}
\newcommand{\ebx}{\end{boxed}}
\newcommand{\bmx}{\left(\begin{array}}
\newcommand{\emx}{\end{array}\right)}
\newcommand{\barra}[1]{\overline{#1}}
\newcommand{\h}[1]{#1^\dagger}
\newcommand{\abs}[1]{\vert #1 \vert}
\newcommand{\ac}[2]{\left\{ #1 , #2 \right\}}
\newcommand{\cc}[2]{\left[ #1 , #2 \right]}
\newcommand{\lvec}[1]{\overleftarrow{#1}}
\newcommand{\rvec}[1]{\overrightarrow{#1}}
\newcommand{\dvec}[1]{\overleftrightarrow{#1}}
\newcommand{\chula}[1]{\mathcal{#1}}
\newcommand{\slx}[1]{ #1\!\!\!/ }
\newcommand{\derp}[2]{\frac{\partial #1}{\partial #2}}
\newcommand{\pdp}[1]{\footnotemark\footnotetext{#1}}


%opening
\title{Fermións de Majorana}
\author{Gonzalo Martínez Lema}
\date{30 / 01 / 13}

\begin{document}

\frame{\titlepage}


\begin{frame}%[allowframebreaks]
\tableofcontents
\end{frame}

\section{Introdución}

\begin{frame} \frametitle{Introdución}
\bi

\item Os neutrinos son fermións, logo cumprirán a ecuación de Dirac:

\beq
(i \slx\partial - m) \nu = 0 \nonumber
\eeq

\item A aproximación habitual consiste en tomar $m_\nu \approx 0$

\item No lagranxiano aparece un termo de masa que determina as propias masas, as mesturas de sabores e a súa natureza

\ei
\end{frame}

\section{Termo de masa de Dirac}

\begin{frame} \frametitle{Termo de Masa de Dirac}
\bi

\item O termo de masa de Dirac ten a seguinte forma para tres sabores:

\beq
L_{masa}^D = -\barra{\rvec\nu_L} M^D \rvec\nu_R + h.c. = \sum_{l'l} \barra\nu_{l'L} \, M_{l'l}^D \, \nu_{lR}+h.c. \nonumber
\eeq

\item Este lagranxiano é invariante baixo U(1):

\beq
\nu \rightarrow e^{i\phi} \nu \quad \Rightarrow \quad \chula L_{masa}^{D'} = \chula L_{masa}^D \nonumber
\eeq

\item Baixo a diagonalización da matriz $M^D = \h U m V$ obtéñense os campos mixtos de sabor que entran nas correntes neutras e cargadas do modelo estándar:

\beq
\nu_{lL} = \sum_{k=1}^{3} U_{lk} \, \nu_{kL} \qquad\qquad \nu_{lR} = \sum_{k=1}^{3} V_{lk} \, \nu_{kR}
\qquad \qquad l = e, \mu, \tau \nonumber
\eeq

\ei

\end{frame}

\section{Teoría de Majorana}

\begin{frame}[allowframebreaks] \frametitle{Teoría de Majorana}
\bi

\item Se no lagranxiano de Dirac facemos a descomposición en campos quirais $\psi = \psi_L + \psi_R$ chégase as ecuacións para estes campos:

\beq
\bc
& i \slx\partial \psi_L = m \psi_R \\
& i \slx\partial \psi_R = m \psi_L
\ec \nonumber
\eeq

\item Estas ecuacións están acopladas pola masa $m$ da partícula. Se son neutrinos e tomamos a masa destes como nula obtéñense as ecuacións de Weyl e a solución son os espinores de Weyl (dúas compoñentes).

\item Pódese manter a descripción de espinores de dúas compoñentes para partículas masivas?

\item Si! Pero isto impón unha restricción: ambas ecuacións deben estar ligadas. Facendo transformacións obtense que esta condición é (condición de Majorana)\pdp{Outra forma de expresar esta relación é $\psi = \psi^C$\\}:

\beq
\psi_R = \psi_L^C = \chula C {\barra\psi_L}^T \qquad \qquad \chula C = i \gamma^2 \gamma^0 \nonumber
\eeq

\item De aquí chégase facilmente a ecuación de Majorana para o campo quiral $\psi_L$:

\beq
\bbx{ i \slx\partial \psi_L = m \chula C \barra\psi_L^T = m \psi_L^C }\ebx \nonumber
\eeq

\item A condición de Majorana implica que o fermión descrito debe ser a súa propia antipartícula $\Rightarrow$ Só se poden describir partículas neutras $\Rightarrow$ Os únicos fermións elementais neutros que coñecemos son os neutrinos. É sinxelo comprobar que baixo a condición de Majorana $\bbx{j^\mu = 0}\ebx$.

\item Outra restricción dun fermión de Majorana aparece na descripción dos espinores: só pode haber dúas compoñentes independentes. Pódense expresar mediante as dúas expresións equivalentes:

\beq
\psi = \bmx{c} \chi \\ -i \sigma^2 \chi^* \emx \qquad \text{ou} \qquad \psi = \bmx{c} i\sigma^2 \omega^* \\ \omega \emx \nonumber
\eeq

En consecuencia o formalismo é máis simple que o de Dirac, pero a fenomenoloxía é distinta no caso de neutrinos masivos.

\ei
\end{frame}

\subsection{Termo de masa de Majorana}

\begin{frame}[allowframebreaks] \frametitle{Termo de masa de Majorana}
\bi

\item Cómo afecta a condición de Majorana ao lagranxiano? Operando cos campos quirais e aplicando a condición de Majorana obtense\pdp{Introdúcese o factor $\frac{1}{2}$ no lagranxiano debido a que $\nu_L$ e $\nu_L^C$\\ non son independentes}:

\beq
\bbx{
\chula L_{masa}^M = -\frac{1}{2} m \barra{\nu_L^C} \nu_L - \frac{1}{2} m \nu_L^C \barra\nu_L = -\frac{1}{2} m \barra{\nu_L^C} \nu_L + h.c.
}\ebx \nonumber
\eeq

\item Aplicando Euler - Lagrange obtense a ecuación de campo de Majorana de novo:

\beq
\bbx{
i\slx\partial \nu_L = m \nu_L^C = m \chula C \barra\nu_L^T
}\ebx \nonumber
\eeq

\item En base á definición $\nu = \nu_L + \nu_L^C$ pódese reescribir o lagranxiano total como:

\beq
\bbx{ \chula L^M = \frac{1}{2} \barra\nu \left( \frac{i}{2} \dvec{\slx\partial} -m\right) \nu }\ebx \nonumber
\eeq

\item O factor $\frac{1}{2}$ diferencia este lagranxiano do de Dirac. Aquí as propiedades de anticonmutación dos campo fermiónicos son esenciais, pois de non selo, o termo de masa do lagranxiano sería nulo.
\ei
\end{frame}

\subsection{Graos de liberdade}

\begin{frame}[allowframebreaks] \frametitle{Graos de liberdade}
\bi

\item Os lagranxianos de Dirac e Majorana son invariantes baixo CPT e transformacións de Lorentz. Isto deixa uns graos de liberdade que se poden determinar. A aplicación de C, P e T segue as seguintes regras:

\beq \small
P\bc t \rightarrow t \\ \rvec r \rightarrow -\rvec r \\  \rvec p \rightarrow -\rvec p \\  \rvec L \rightarrow \rvec L \\  \rvec s \rightarrow \rvec s \\ h \rightarrow -h \\  \nu \rightarrow \nu \ec \qquad
T\bc t \rightarrow -t \\ \rvec r \rightarrow \rvec r \\  \rvec p \rightarrow -\rvec p \\  \rvec L \rightarrow -\rvec L \\  \rvec s \rightarrow -\rvec s \\ h \rightarrow h \\  \nu \rightarrow \nu \ec \qquad
C\bc t \rightarrow t \\ \rvec r \rightarrow \rvec r \\  \rvec p \rightarrow \rvec p \\  \rvec L \rightarrow \rvec L \\  \rvec s \rightarrow \rvec s \\ h \rightarrow h \\  \nu \rightarrow \barra\nu \ec \nonumber
\eeq

\item Facendo tódalas transformacións posibles nun campo de Dirac obtense o seguinte esquema:

\begin{figure}[h!]
	\centering
		\includegraphics[scale=.3]{TDirac.jpg}
	\caption{ Posibles transformacións sobre un campo de Dirac}
\end{figure}

\item Para obter o caso de Majorana basta facer $\barra\nu \rightarrow \nu$:

\begin{figure}[h!]
	\centering
		\includegraphics[scale=.4]{TMajorana.jpg}
	\caption{ Posibles transformacións sobre un campo de Majorana}
\end{figure}

\item Isto dinos que no formalismo de Dirac hai 4 graos de liberdade:

\beq
\nu (\rvec p , h) \qquad \nu (\rvec p , -h) \qquad \barra\nu (\rvec p , h) \qquad \barra\nu (\rvec p , -h) \nonumber
\eeq

\item Mentres que no de Majorana só hai 2:

\beq
\nu ( \rvec p , h) \qquad \qquad \nu ( \rvec p , -h ) \nonumber
\eeq

\ei
\end{frame}

\subsection{Número leptónico}

\begin{frame} \frametitle{Número leptónico}
\bi

\item Ao dicir que un neutrino é o mesmo que un antineutrino, o número leptónico perde sentido pois:

\beq
 L_\nu = 1 \qquad L_{\barra\nu}=-1 \qquad -1 \neq 1 \quad \forall \quad 1 \neq 0 \nonumber
\eeq

\item Debido a que a masa do neutrino é ínfima defínese un número leptónico efectivo en base á invariancia de $\chula L$ baixo U(1) resultando:

\beq
\bc
h=-1 \rightarrow L_{ef} = +1 \\
h=+1 \rightarrow L_{ef} = -1
\ec \nonumber
\eeq

\item A pesar que de neutrino $\equiv$ antineutrino, distínguense pola súa helicidade
\ei
\end{frame}

\subsection{Formalismo de dúas compoñentes}

\begin{frame} \frametitle{Formalismo de dúas compoñentes}
\bi

\item Representando os espinores de Majorana como:

\beq
\nu_L = \bmx{c} \chi \\ 0 \emx \qquad \nu_L^C = \bmx{c} 0 \\ \omega \emx \qquad \rightarrow \qquad \nu = \bmx{c} \chi \\ \omega \emx \nonumber
\eeq

Obtense facilmente o lagranxiano en función dos espinores:

\beq
\chula L^M = \frac{-i}{2} \left[ \h\omega \tilde\sigma^\mu \dvec{\slx\partial} \omega + m \left( \omega^T \sigma^2 \omega  \h\omega \sigma^2 \omega^* \right) \right] \nonumber
\eeq

E ao aplicar Euler Lagrange obtense a ecuación de campo de Majorana dun campo de dúas compoñentes:

\beq
\tilde\sigma^\mu \partial_\mu \omega - m \sigma^2 \omega^* = 0 = \left( \partial_0 - \rvec\sigma \cdot \rvec\nabla \right) \omega + im \sigma^2 \omega^* \nonumber
\eeq

\ei
\end{frame}

\subsection{Masa de Majorana efectiva}

\begin{frame} \frametitle{Masa de Majorana efectiva}
\bi

\item Non se pode construir un lagranxiano compatible co modelo estándar e co formalismo de Majorana simultaneamente debido a que non é renormalizable.

\item Pódese construir un lagranxiano de dimensión 5 que debido á ruptura de simetría electrofeble xera un termo de masa similar ao de Majorana. Comparándoo con este obtense que a masa dun fermión de Majorana será da orde:

\beq
m \propto \frac{m_D^2}{M} \nonumber
\eeq

\ei
\end{frame}

\section{Mestura de tres neutrinos de Majorana}

\begin{frame}[allowframebreaks] \frametitle{Mestura de tres neutrinos de Majorana}
\bi

\item Tomando as tres xeracións de neutrinos temos o vector de campos levóxiros:

\beq
\rvec{\nu_L'} = \bmx{ccc} \nu_{eL}' & \nu_{\mu L}' & \nu_{\tau L}' \emx^T \nonumber
\eeq

\item Con isto o lagranxiano adopta a forma:

\beq
\chula L_{masa}^M= -\frac{1}{2} \barra{\rvec\nu_L} M^M {\rvec\nu_L}^C +h.c. = -\frac{1}{2} \sum_{l,l' = e,\mu,\tau} \barra\nu_{l'L} M_{l'l}^M \nu_{l' L}^C + h.c. \nonumber
\eeq

\item $M^M$ é unha matriz simétrica que se diagonaliza mediante $M^M = U m U^T$ e de aquí pódese obter a relación entre os campos de sabor do modelo estándar e os de Majorana:

\beq
\bbx{
\nu_{lL} = \sum_{k=1}^3 U_{lk} \nu_{lK}
}\ebx \nonumber
\eeq

E o lagranxiano quedaría:

\beq
\bbx{
\chula L = \frac{1}{2} \sum_k \barra\nu_k \left( i\slx\partial - m_k \right) \nu_k \nonumber
}\ebx
\eeq

\ei
\end{frame}

\section{Dobre $\beta$ sen neutrinos}

\begin{frame} \frametitle{Dobre $\beta$ sen neutrinos}
\bi

\item Debido a que un neutrino é igual a un antineutrino poderíamos ter que nunha desintegración $\beta$ un neutrino pase a ser un antineutrino desencadeando outra $\beta$.

\item A $\beta\beta$ é un proceso único logo o cambio de helicidade necesario para que se produza debe ser virtual. O diagrama de Feynman correspondente é o seguinte:

\begin{figure}[h!]
	\centering
		\includegraphics[scale=.5]{Dobre.jpg}
	\caption{Diagrama de Feynman para a desintegración dobre $\beta$ sen neutrinos}
\end{figure}

\ei
\end{frame}

\section{Termo de masa de Dirac e Majorana}

\begin{frame} \frametitle{Termo de masa de Dirac e Majorana}
\bi

\item Pódese construir un lagranxiano que leve simultaneamente o termo de masa de Dirac e o de Majorana. Escrito en forma matricial sería:

\beq
\bbx{
\chula L_{masa}^{D+M} = - \frac{1}{2} \barra n_L M^{M+D} n_L^C + h.c.
}\ebx \nonumber
\eeq

Onde:

\beq
n_L = \bmx{c} \nu_L \\ \nu_R^C \emx \qquad M^{M+D} = \bmx{cc} M_L^M & M^D \\ {M^D}^T & M_R^M \emx \nonumber
\eeq

\item Ao diagonalizar a matriz obtéñense os autoestados da masa:

\beq
\small
\bbx{ m_{1,2}' = \frac{m_L + m_R \mp \sqrt{(m_R - m_L)^2 + 4m_D^2}}{2} = \eta_{1,2} \abs{m_{1,2}'} = \eta_{1,2} m_{1,2} }\ebx \nonumber
\eeq

\ei
\end{frame}

\subsection{Mecanismo do balancín}

\begin{frame} \frametitle{Mecanismo do balancín}
\bi

\item É a teoría máis prometedora que explica a ínfima masa dos neutrinos. Baséase no termo de masa de Dirac e Majorana. Pártese de tres suposicións:
\bi

\item $m_L = 0$

\item $m_D \approx$ masa dun quark/leptón da familia

\item $m_R \equiv M_R \gg m_D$

\ei

\item Con estas condicións as masas que saen son:

\beq
m_{1,2} \approx
\bc
\frac{m_D^2}{M_R} \ll m_D \\
M_R \gg m_D
\ec \nonumber
\eeq

\item E finalmente para tres familias tense:

\bea
m_\nu = m_\ll &\approx& - m_D M_R^{-1} m_D^T \nonumber \\
m_\gg &\approx& M_R \nonumber
\eea

\ei
\end{frame}

\section{Conclusións}

\begin{frame} \frametitle{Conclusións}
\bi

\item Se o mecanismo do balancín se dá na natureza:

\bi

\item Os neutrinos son partículas de Majorana

\item As masas son moitísimo menores que as dos quarks e leptóns

\item Deben existir partículas de Majorana pesadas

\ei

\item Se se dá a condición de Majorana, os neutrinos son a súa propia antipartícula.

\item A conservación do número leptónico depende da natureza dos neutrinos.

\item Se as matrices de Dirac ou Majorana non son diagonais mestúranse distintos campos de sabor

\item A observación experimental da $\beta\beta$ sería unha confirmación experimental de que os neutrinos son partículas de Majorana

\ei
\end{frame}

\begin{frame} \frametitle{Bibliografía}

\begin{thebibliography}{4}
\bibitem{libro1}
Fundamentals of neutrino physics and astrophysics - \textit{Giunti \& Kim}

\bibitem{libro2}
Introductory to the physics of massive and mixed neutrinos  - \textit{Bilenky}

\bibitem{libro3}
Gauge theory of weak interactions - \textit{Greiner}

\bibitem{libro4}
Gauge theory of elementary particle physics - \textit{Cheng - Li}

\end{thebibliography}

\end{frame}

\end{document}








