%GALILEO667.NET | AISNE
%Curso LaTeX Ejercicio 4 - Tablas y Matrices

\documentclass[a4paper,10pt]{article}
\usepackage[spanish]{babel}
\usepackage[utf8]{inputenc}
\usepackage{bm} %letras griegas en negrita (uso: \bm{\alpha})
\usepackage{graphicx}
\usepackage{amsmath,amsfonts,latexsym,cancel,textcomp,anysize,color,eurosym,amsthm}
\marginsize{1.5cm}{1cm}{1.5cm}{1.5cm} %izq,der,sup,inf
\parindent=0mm %sangría
\parskip=3mm %espacio entre párrafos
\usepackage{fancyhdr}  
\newcommand{\bi}{\begin{itemize}}
\newcommand{\ei}{\end{itemize}}
\newcommand{\newint}[2]{$\displaystyle \int_0^{\pi} \ln(#1)\cdot \cos(#1) d#2$}
\newtheorem{coro}{Corolario Importante:}

\begin{document}

\begin{center}
\textsc{ATENCIÓN, PARA RESOLVER ESTE EJERCICIO HAY QUE COPIAR LO QUE ESTÁ ESCRITO EN ESTE DOCUMENTO, PERO TODAS
LAS ÓRDENES DEBEN DE HACERSE CON COMANDOS NUEVOS MEDIANTE NEWCOMMAND.} 
\end{center}



Teneis que renombrar las ordenes de creación de listas para que al introducir:
\begin{verbatim}
\bi
\item Hola
\item Buenas
\ei
\end{verbatim}

Se imprima:
\bi
\item Hola
\item Buenas
\ei

\vspace{2cm}

Teneis que crear un nuevo comando para que cuando introduzcais:
\begin{verbatim}
\newint{5}{x}
\end{verbatim}
Se imprima:
\newint{5}{x}

\vspace{2cm}


Teneis que usar newtheorem para imprimir lo siguiente:
\begin{coro}
Bla bla bla bla
\end{coro}



\end{document}



  
