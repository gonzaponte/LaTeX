\documentclass[a4paper,10pt]{article}
\usepackage[spanish]{babel}
\usepackage[utf8]{inputenc}
\usepackage{bm} %letras griegas en negrita (uso: \bm{\alpha})
\usepackage{graphicx}
\usepackage{amsmath,amsfonts,latexsym,cancel,textcomp,anysize,color}
\marginsize{1.5cm}{1cm}{1.5cm}{1.5cm} %izq,der,sup,inf
\parindent=0mm %sangría
\parskip=3mm %espacio entre párrafos


%opening
\title{}
\author{}
\date{}

\begin{document}

\section{Ecuaciones Importantes}

\subsection{El teorema de Bernoulli}
El teorema de Bernoulli es un teorema fundamental de la \textit{Hidrodinámica}. Es un ``Teorema de Conservación'' que
explica el comportamiento de un flujo laminar:

\begin{equation}
\frac{V^2 \rho}{2}+P+\rho\cdot g \cdot h=\mathrm{constante}
\end{equation}

\subsection{El teorema de Euler}

El teorema de Euler $e^{i \theta}=\cos\theta+i \sen\theta$ es un teorema trigonométrico básico.

\subsection{Electomagnetismo}

Estas son algunas de las ecuaciones de determinan el comportamiento electromagnético:

$$\nabla\cdot \vec{E}=\frac{\rho}{\epsilon_0} $$
$$\oint \vec{E} \cdot d\vec{S}=\frac{Q_{enc}}{\epsilon_0} $$
$$\nabla \vec{B}=-\frac{\partial \vec{E}}{\partial t}$$

\subsection{Una integral sencillita}

\begin{equation}
\int\limits_1^2 \frac{1}{x} dx=\log_e(2)=\ln(2)\approx 0.69
\end{equation}



\end{document}



  
