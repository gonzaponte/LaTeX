\documentclass[a4paper,10pt]{article}
\usepackage[spanish]{babel}
\usepackage[utf8]{inputenc}
\usepackage{bm} %letras griegas en negrita (uso: \bm{\alpha})
\usepackage{graphicx}
\usepackage{amsmath,amsfonts,latexsym,cancel,textcomp,anysize,color}
\marginsize{1.5cm}{1cm}{1.5cm}{1.5cm} %izq,der,sup,inf
\parindent=0mm %sangría
\parskip=3mm %espacio entre párrafos


%opening
\title{Imágenes en \LaTeX}
\author{Pepito Grillo}
\date{Noviembre de 2012}

\begin{document}

\maketitle

\begin{center}
\textsc{En este ejercicio se pueden sustituir las imágenes y los captions por los que desee el alumno.}
\end{center}

En este apartado hemos aprendido a añadir imágenes con caption y numeración automática:

\begin{figure}[h!]
\begin{center}
\includegraphics[scale=0.35] %Tamaño
{./images/acdc.jpg}  %descripción de la imagen
\caption{AC/DC Es el mejor grupo de Rock de la historia.}
\end{center}
\end{figure}


Y a insertar imágenes sin numerar y sin caption, en cualquier parte del documento:

\begin{figure}[h!]
\begin{flushleft}
\includegraphics[scale=0.15] %Tamaño
{./images/harley.jpg}  %descripción de la imagen
\end{flushleft}
\end{figure}


\end{document}



  
