%GALILEO667.NET | AISNE
%Curso LaTeX Ejercicio 4 - Tablas y Matrices

\documentclass[a4paper,10pt]{article}
\usepackage[spanish]{babel}
\usepackage[utf8]{inputenc}
\usepackage{bm} %letras griegas en negrita (uso: \bm{\alpha})
\usepackage{graphicx}
\usepackage{amsmath,amsfonts,latexsym,cancel,textcomp,anysize,color,eurosym}
\marginsize{1.5cm}{1cm}{1.5cm}{1.5cm} %izq,der,sup,inf
\parindent=0mm %sangría
\parskip=3mm %espacio entre párrafos


%opening
\title{Imágenes en \LaTeX}
\author{Pepito Grillo}
\date{Noviembre de 2012}

\begin{document}


Vamos a hacer una tabla sencilla con bordes, que esté numerada y con una explicación:

\begin{table}[h!]
\begin{center}
\begin{tabular}{|c|l|}
\hline
\textbf{planeta} & \textbf{Masa} \\
\hline
Mercurio & 0.06 umt\\
\hline
Venus & 0.82 umt\\
\hline
Tierra & 1 umt\\
\hline
Marte & 0.11 umt\\
\hline
Júpiter & 318 umt\\
\hline
\end{tabular}
\caption{Algunos planetas del SS y su masa en unidades de masa terrestre.}
\end{center}
\end{table}


Vamos a hacer ahora una tabla sin bordes y sin numeración alguna:

\begin{center}
Salario Mínimo interprofesional:\\
\begin{tabular}{cl}
Francia & 1309\geneuro \\
Reino Unido & 1190\geneuro \\
Bélgica & 1283\geneuro \\
Austria & 1000\geneuro \\
Grecia & 680\geneuro \\
España & 600\geneuro \\
\end{tabular}
\end{center}

Vamos a insertar una matriz normalita:

$$
M = \left( \begin{array}{cccc} %Número de columnas y justificado junto al left seleccionar (,[, |...
1 & 2 & 3 & 4\\
5 & 6 & 7 & 8\\
9 & 10 & 11 & 12\\
13 & 14 & 15 & 16  \end{array} \right) 
$$


Ahora dos, cambiando los corchetes:


$$
M = \left[ \begin{array}{ccc} %Número de columnas y justificado junto al left seleccionar (,[, |...
1 & 2 & 3 \\
4 & 5 & 6 \\
7 & 8 & 9 \end{array} \right]
\cdot
\left[ \begin{array}{c} %Número de columnas y justificado junto al left seleccionar (,[, |...
b1 \\
b2 \\
b3 \end{array} \right]
$$

Analicemos el campo electrostático de una esfera uniformemente cargada:

$$
\begin{array}{c}
\text{Campo Eléctrico}
\end{array}  = 
\left\{ \begin{array}{cc} %Número de columnas y justificado junto al left seleccionar (,[, |...
r\leq a : & \vec{E}=\frac{Qr}{4\pi\epsilon_0 a^3}\hat r   \\
r\neq a : & \vec{E}=\frac{Q}{4\pi\epsilon_0 r^2}\hat r     \end{array} \right. 
$$

\end{document}



  
